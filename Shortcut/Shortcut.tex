\documentclass[UTF8]{ctexart}
\usepackage{amsmath}

\makeatletter %使\section中的内容左对齐
\renewcommand{\section}{\@startsection{section}{1}{0mm}
  {-\baselineskip}{0.5\baselineskip}{\bf\leftline}}
\makeatother

\title{开发笔记}
\author{Ivan Lin}
\date{\today}
\begin{document}
\maketitle
\section*{Visual Studio}
\noindent Resharper插件\\
alt + o: .h和.cpp文件切换\\
alt + 鼠标: 框选模式\\
ctrl+k + ctrl+c: 注释代码\\
\\\section*{计算机图形学}
\noindent 坐标系模拟:拇指x,食指y,中指z。左手系和右手系\\
标准化向量 = 单位向量 = 法线,$V_{norm} = \frac{V}{|V|}$\\
\textbf{a} + \textbf{b} 几何解释:\textbf{a}的头连接\textbf{b}的尾,然后从\textbf{a}的尾向\textbf{b}的头画一个向量\\
\textbf{a} - \textbf{b} 几何解释:\textbf{a}的尾连接\textbf{b}的尾,然后从\textbf{b}的头向\textbf{a}的头画一个向量\\
\textbf{a $\cdot$ b} = \textbf{$a_1b_1+...+a_nb_n$}, 点乘,几何解释:\textbf{a $\cdot$ b} = \textbf{|a||b|}cos$\theta$(两向量夹角)\\
\\\section*{Swift}
\noindent \textbf{http://blackblake.synology.me/wordpress/?p=29}: Swift里的Optional和Unwrapping\\
\end{document}