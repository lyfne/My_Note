\documentclass[UTF8]{ctexart}
\usepackage{}
\usepackage{amssymb}
\usepackage{amsmath}

\makeatletter %使\section中的内容左对齐
\renewcommand{\section}{\@startsection{section}{1}{0mm}
  {-\baselineskip}{0.5\baselineskip}{\bf\leftline}}
\makeatother

\title{开发笔记}
\author{Ivan Lin}
\date{\today}
\begin{document}
\maketitle

\section*{Visual Studio}
\noindent Resharper插件\\
alt + o: .h和.cpp文件切换\\
alt + 鼠标: 框选模式\\
ctrl+k + ctrl+c: 注释代码\\
shift+alt+up/down: 框选模式上下\\
ctrl+alt+a: open Command Window\\
ReSharper\_Suspend/ReSharper\_Resume in Command Window: close/open ReSharper\\
\section*{计算机图形学}
\noindent 坐标系模拟:拇指x,食指y,中指z。左手系和右手系\\
标准化向量 = 单位向量 = 法线,$\textbf{v}_{norm} = \frac{\textbf{v}}{\textbf{\textbar v\textbar }}$\\
\textbf{a} + \textbf{b} 几何解释:\textbf{a}的头连接\textbf{b}的尾,然后从\textbf{a}的尾向\textbf{b} 的头画一个向量\\
\textbf{a} - \textbf{b} 几何解释:\textbf{a}的尾连接\textbf{b}的尾,然后从\textbf{b}的头向\textbf{a} 的头画一个向量\\
向量点乘:\textbf{a $\cdot$ b}(\textbf{ab}) = \textbf{$a_1b_1+...+a_nb_n$}, 几何解释:\textbf{a $\cdot$ b} = \textbf{\textbar a\textbar \textbar b\textbar }cos$\theta$(两向量夹角)\\
向量投影:\textbf{v}分解为平行和垂直于\textbf{n}的两个分量。\\
\[ \textbf{v}_{||} = \textbf{n}\frac{\textbf{v $\cdot$ n}}{\textbf{\textbar n\textbar }^2} \qquad \textbf{v}_{\bot} = \textbf{\textbar v\textbar } - \textbf{v}_{||}\]
向量叉乘:仅可用于3D向量,$\textbf{a}\times\textbf{b} = \begin{bmatrix} \textbf{a}_y\textbf{b}_z - \textbf{a}_z\textbf{b}_y \\ \textbf{a}_z\textbf{b}_x - \textbf{a}_x\textbf{b}_z \\ \textbf{a}_x\textbf{b}_y - \textbf{a}_y\textbf{b}_x \end{bmatrix}$,几何解释:结果向量垂直于原来两个向量,$|\textbf{a}\times\textbf{b}| = |\textbf{a}||\textbf{b}|sin\theta$,$|\textbf{a}\times\textbf{b}| = 0$表示\textbf{a}与\textbf{b}平行或有一个为\textbf{0}\\
矩阵转置: \textbf{$M^T$}, 其列由\textbf{M}的行组成,\textbf{${M^T}_{ji}$} = \textbf{$M_{ij}$}\\
\textbf{${(AB)}^T = {B^T}{A^T}$}, 可推广到字符串翻转\\
$P_{camera} = P_{object}M_{object\to world}M_{world\to camera}$\\
线性变换: F(a+b) = F(a)+F(b), F(ka) = kF(a), 则称映射F是线性的(\textbf{aM}满足此条件)\\
仿射变换: 线性变换后接平移, $v^{'} = v\textbf{M} + \textbf{b}$\\
对\textbf{aM}, 求逆变换等价于求矩阵的逆\\
矩阵行列式: $|\textbf{M}|  = \sum_{j=1}^{n}m_{ij}c_{ij} = \sum_{j=1}^{n}m_{ij}(-1)^{i+j}| \textbf{M}^{\{ij\}}|$\\
矩阵的逆: $M(M^{-1}) = M^-1M = I$, 不可逆矩阵又称奇异矩阵,奇异矩阵行列式为0\\
标准伴随矩阵:adj\textbf{M}, M的代数余子式矩阵的转置矩阵。$M^{-1} = \frac{adj\textbf{M}}{|\textbf{M}|}$\\
正交矩阵:\textbf{$MM^{T} = I$}, 旋转和镜像矩阵是正交矩阵。正交矩阵满足:矩阵的每一行都是单位向量,矩阵的所有行相互垂直。\\
Vector4, 齐次坐标。(x, y, z, w)实际代表3D中的(x/w, y/w, z/w)\\
旋转矩阵:描述一个坐标中基向量到另一个坐标基向量的转换。
欧拉角:heading-pitch-bank约定。
\section*{Sublime Text 2}
\noindent ctrl+shift+up/down: move line up/down\\
ctrl+alt+up/down: block edit up/down\\
\section*{Swift}
\noindent \textbf{http://blackblake.synology.me/wordpress/?p=29}: Swift里的Optional和Unwrapping\\
\section*{PhotoShop}
\noindent alt+ctrl+c: Resize Canvas\\
alt+ctrl+shift+s: Save for web\\
\section*{LaTeX}
\noindent \%!Mode:: "TeX:UTF-8": make WinEdt show Chinese\\
\section*{Git}
\noindent gitk file/folder: show commit with file\\
\section*{Windows}
\noindent 放大镜: ctrl+alt+d: 停靠模式; ctrl+alt+l: 窗口模式; win++: 放大; win+esc: 退出放大镜\\
\section*{JavaScript}
\noindent JavaScript组成: ECMAScript, DOM:针对XML文件的操作接口, BOM:浏览器对象模型,HTML5标准化\\
浮点数误差:0.1 + 0.2 = 0.300000000004,通过x10法解决\\
\end{document}
